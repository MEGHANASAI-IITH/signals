\documentclass[journal,12pt,onecolumn]{IEEEtran}
\usepackage{cite}
\usepackage{amsmath,amssymb,amsfonts,amsthm}
\usepackage{algorithmic}
\usepackage{graphicx}
\usepackage{textcomp}
\usepackage{xcolor}
\usepackage{txfonts}
\usepackage{listings}
\usepackage{enumitem}
\usepackage{mathtools}
\usepackage{gensymb}
\usepackage{comment}
\usepackage[breaklinks=true]{hyperref}
\usepackage{tkz-euclide}
\usepackage{listings}
\usepackage{gvv}
\def\inputGnumericTable{}
\usepackage[latin1]{inputenc}
\usepackage{color}
\usepackage{array}
\usepackage{longtable}
\usepackage{calc}
\usepackage{multirow}
\usepackage{hhline}
\usepackage{ifthen}
\usepackage{lscape}

\newtheorem{theorem}{Theorem}[section]
\newtheorem{problem}{Problem}
\newtheorem{proposition}{Proposition}[section]
\newtheorem{lemma}{Lemma}[section]
\newtheorem{corollary}[theorem]{Corollary}
\newtheorem{example}{Example}[section]
\newtheorem{definition}[problem]{Definition}
\newcommand{\BEQA}{\begin{eqnarray}}
    \newcommand{\EEQA}{\end{eqnarray}}
\newcommand{\define}{\stackrel{\triangle}{=}}
\theoremstyle{remark}
\newtheorem{rem}{Remark}

\begin{document}
    
    \bibliographystyle{IEEEtran}
    \vspace{3cm}
    
    \title{Gate 2022 EC Q55}
    \author{EE23BTECH11212 - Manugunta Meghana Sai$^{*}$% <-this % stops a space
    }
    \maketitle
    \bigskip
    
    \renewcommand{\thefigure}{\theenumi}
    \renewcommand{\thetable}{\theenumi}
    
    \vspace{3cm}
    \textbf{Gate 2022 EE Q55} 
    
    For a vector $\bar{x} = [x[0], x[1], \dots, x[7] ]$, the $8$-point discrete Fourier transform (DFT) is denoted by $\bar{X} = \text{DFT}(\bar{x}) = [X[0],X[1],\dots,X[7]]$, where
    \begin{align*}
    X[k] = \sum_{n=0}^{7}x[n]\exp\left(-j\frac{2\pi}{8}nk\right).
    \end{align*} 
    Here $j = \sqrt{-1}$. If $\bar{x} = [1,0,0,0,2,0,0,0]$ and $\bar{y} = \text{DFT}(\text{DFT}(\bar{x}))$, then the value of $y[0]$ is\\
    \solution
    DFT of $\bar{x}$
    \\For $k=0$ :
    \begin{align}
    X[0] &= \sum_{n=0}^{7}x[n]\\
    &= x[0] + x[1] + \dots + x[7]\\
    &= 1 + 0 + 0 + 0 + 2 + 0 + 0 + 0\\
    &= 3
    \end{align}
    For $k=1$ :
    \begin{align}
    X[1] &= \sum_{n=0}^{7}x[n]\exp\left(-j\frac{2\pi}{8}n\right)\\
    &= x[0] + x[1]\exp\left(-j\frac{2\pi}{8}\right) + x[2]\exp\left(-j\frac{4\pi}{8}\right) + \dots + x[7]\exp\left(-j\frac{14\pi}{8}\right)\\
    &= 1 - 2  \\
    &= -1
    \end{align}
    For $k=2$ :
    \begin{align}
    X[2] &= \sum_{n=0}^{7}x[n]\exp\left(-j\frac{2\pi}{8}2n\right)\\
    &= x[0] + x[1]\exp\left(-j\frac{4\pi}{8}\right) + x[2]\exp\left(-j\frac{8\pi}{8}\right) + \dots + x[7]\exp\left(-j\frac{28\pi}{8}\right)\\
    &= 1 + 2  \\
    &= 3
    \end{align}
    For $k=3$ :
    \begin{align}
    X[3] &= \sum_{n=0}^{7}x[n]\exp\left(-j\frac{2\pi}{8}3n\right)\\
    &= x[0] + x[1]\exp\left(-j\frac{6\pi}{8}\right) + x[2]\exp\left(-j\frac{12\pi}{8}\right) + \dots + x[7]\exp\left(-j\frac{42\pi}{8}\right)\\
    &= 1 - 2  \\
    &= -1
    \end{align}
    For $k=4$ :
    \begin{align}
    X[4] &= \sum_{n=0}^{7}x[n]\exp\left(-j\frac{2\pi}{8}4n\right)\\
    &= x[0] + x[1]\exp\left(-j\frac{8\pi}{8}\right) + x[2]\exp\left(-j\frac{16\pi}{8}\right) + \dots + x[7]\exp\left(-j\frac{56\pi}{8}\right)\\
    &= 1 + 2  \\
    &= 3
    \end{align}
    For $k=5$ :
    \begin{align}
    X[5] &= \sum_{n=0}^{7}x[n]\exp\left(-j\frac{2\pi}{8}5n\right)\\
    &= x[0] + x[1]\exp\left(-j\frac{10\pi}{8}\right) + x[2]\exp\left(-j\frac{20\pi}{8}\right) + \dots + x[7]\exp\left(-j\frac{70\pi}{8}\right)\\
    &= 1 - 2  \\
    &= -1
    \end{align}
    For $k=6$ :
    \begin{align}
    X[6] &= \sum_{n=0}^{7}x[n]\exp\left(-j\frac{2\pi}{8}6n\right)\\
    &= x[0] + x[1]\exp\left(-j\frac{12\pi}{8}\right) + x[2]\exp\left(-j\frac{24\pi}{8}\right) + \dots + x[7]\exp\left(-j\frac{84\pi}{8}\right)\\
    &= 1 + 2  \\
    &= 3
    \end{align}
    For $k=7$ :
    \begin{align}
    X[7] &= \sum_{n=0}^{7}x[n]\exp\left(-j\frac{2\pi}{8}7n\right)\\
    &= x[0] + x[1]\exp\left(-j\frac{14\pi}{8}\right) + x[2]\exp\left(-j\frac{28\pi}{8}\right) + \dots + x[7]\exp\left(-j\frac{98\pi}{8}\right)\\
    &= 1 - 2  \\
    &= -1
    \end{align}
    \begin{align}
    \bar{X} &= \text{DFT}(\bar{x}) = [X[0],X[1],\dots,X[7]]\\
    \bar{X} &= [3,-1,3,-1,3,-1,3,-1]
    \end{align}
    \begin{align}
    \bar{y} &= \text{DFT}(\text{DFT}(\bar{x}))\\
    \bar{y} &= [3,-1,3,-1,3,-1,3,-1]\\
    y[0] &= \sum_{n=0}^{7}x[n]\\
    &= x[0] + x[1] + \dots + x[7]\\
    &= 3 -1 +3 -1 +3 -1 +3 -1 = 8
    \end{align}
\end{document}

