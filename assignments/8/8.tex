
\documentclass[journal,12pt,onecolumn]{IEEEtran}
\usepackage{cite}
\usepackage{amsmath,amssymb,amsfonts,amsthm}
\usepackage{algorithmic}
\usepackage{graphicx}
\usepackage{textcomp}
\usepackage{xcolor}
\usepackage{txfonts}
\usepackage{listings}
\usepackage{enumitem}
\usepackage{mathtools}
\usepackage{gensymb}
\usepackage{comment}
\usepackage[breaklinks=true]{hyperref}
\usepackage{tkz-euclide}
\usepackage{listings}
\usepackage{gvv}
\def\inputGnumericTable{}
\usepackage[latin1]{inputenc}
\usepackage{color}
\usepackage{array}
\usepackage{longtable}
\usepackage{calc}
\usepackage{multirow}
\usepackage{hhline}
\usepackage{ifthen}
\usepackage{lscape}

\newtheorem{theorem}{Theorem}[section]
\newtheorem{problem}{Problem}
\newtheorem{proposition}{Proposition}[section]
\newtheorem{lemma}{Lemma}[section]
\newtheorem{corollary}[theorem]{Corollary}
\newtheorem{example}{Example}[section]
\newtheorem{definition}[problem]{Definition}
\newcommand{\BEQA}{\begin{eqnarray}}
    \newcommand{\EEQA}{\end{eqnarray}}
\newcommand{\define}{\stackrel{\triangle}{=}}
\theoremstyle{remark}
\newtheorem{rem}{Remark}

\begin{document}
    
    \bibliographystyle{IEEEtran}
    \vspace{3cm}
    
    \title{Gate 2021 BM Q8}
    \author{EE23BTECH11212 - Manugunta Meghana Sai$^{*}$% <-this % stops a space
    }
    \maketitle
    \bigskip
    
    \renewcommand{\thefigure}{\theenumi}
    \renewcommand{\thetable}{\theenumi}
    
    \vspace{3cm}
    
    For a linear stable second order system, if the unit step response is such that peak time is twice the rise time, then the system is . 
    \begin{enumerate}
    \item underdamped\\
    \item undamped\\
    \item overdamped\\
    \item critically damped\\
    \end{enumerate}
    \solution
    \begin{table}[h!]
 	\centering
 	\resizebox{6 cm}{!}{
 		\begin{table}[H]
    \centering
    \renewcommand\thetable{1}
    \setlength{\extrarowheight}{9pt}
    \resizebox{0.51\textwidth}{!}{
    \begin{tabular}{|c|c|c|}
    \hline
    \textbf{$r\brak{i}$} & \textbf{$p\brak{i}$} & \textbf{$k\brak{i}$} \\ \hline
    $0.14344781+0.j$ &0.86678844+0.j&$-0.00041905$  \\ \hline
    $-0.07135684-0.04078045j$ &0.92424846+0.11481887j&$-$  \\ \hline
    $-0.07135684+0.04078045j$ &0.92424846-0.11481887j&$-$  \\ \hline
    \end{tabular}}
    \caption{Values of $ r(i) , p(i) , k(i)$}
    \label{tab:values of r(i) , p(i) , k(i)}
    \end{table}

 	}
 	\caption{Given Parameters}
 	\label{tab:msmBMgate8tab1}
     \end{table} 
    \\The rise time is given by:
    \begin{align}
    t_{r} = \frac{\pi-\theta}{\omega_{n} \sqrt{1-\zeta^{2}}}
    \end{align}
    The peak time is given by:
    \begin{align}
    t_{p} = \frac{\pi}{\omega_{n} \sqrt{1-\zeta^{2}}}
    \end{align}
    as, peak time is twice the rise time:
    \begin{align}
    t_{p} &= 2t_{r}\\
    \frac{\pi}{\omega_{n} \sqrt{1-\zeta^{2}}} &= 2\frac{\pi-\theta}{\omega_{n} \sqrt{1-\zeta^{2}}}\\
    \theta &= \frac{\pi}{2}
    \end{align}
    as, $\theta = \frac{\pi}{2}$, both roots of the system are imaginary, so 
    \begin{align}
    G\brak{s} = \frac{\omega_n^2}{s^2 + 2\zeta\omega_n s + \omega_n^2}
    \end{align}
    So, for the denominator to have two imaginary roots $2\zeta\omega_n$ should be zero.
    so,
    \begin{align}
    \zeta = 0
    \end{align}
    $\zeta$ is zero, hence system is undamped. 
\end{document}
